%! Author = oterk
%! Date = 04/12/2023

% Preamble
\documentclass[11pt]{article}

% Packages
\usepackage{amsmath}
\usepackage{booktabs}
\newcommand*{\prob}[1]{\ensuremath{\mathsf{P}(#1)}}
% Document
\begin{document}

\section*{Report 5: Pacman Ghostbusters}

\section*{Question 6}
\subsection*{Write down the equation of the inference problem you are trying to solve}
The equation we are trying to solve in the function \textbf{observeUpdate()} is:\\
    $ \prob{X \mid Z} = \dfrac{\prob{Z \mid X} \cdot \prob{X}}{\prob{Z}} $

\section*{Question 7}
\subsection*{Write down the equation of the inference problem you are trying to solve}
% P(newPos | oldPos, gameState) = Σ P(newPos | oldPos) * P(oldPos | gameState)
The equation we are trying to solve in the function \textbf{elapseTime()} is:\\

    $ \prob{X \mid Y, Z} = \sum \prob{X \mid Y} \cdot \prob{Y \mid Z} $

Where:\\
\begin{itemize}
\item $ X = newPos $
\item $ Y = oldPos $
\item $ Z = gameState $
\item $ \prob{X \mid Y, Z} $ is the updated belief about the ghost being at position newPos given the old position oldPos and the current game state gameState. This is what we are trying to compute.
\item $ \prob{X \mid Y} $ Is the probability of the ghost moving to position newPos given that it was at position oldPos.
This is obtained from the \textbf{getPositionDistribution} method
\item $ \prob{Y \mid Z} $ is the current belief about the ghost being at position oldPos given the current game state \textbf{gameState}.
This is stored in \textbf{self.beliefs[oldPos]}
\end{itemize}
\section*{Question 8}
\subsection*{Can you think of a better strategy than the greedy strategy? Describe how Pacman can use the probability values to their advantage and more effectively hunt ghosts}
A better strategy than the greedy strategy could be a probabilistic approach that takes into account the belief distributions of the ghosts' positions.
This strategy would not only consider the most likely position of each ghost, but also the probability of the ghost being in each position.
This way, Pacman can make more informed decisions and more effectively hunt the ghosts.
\begin{itemize}
    \item Here is the pseudocode for this strategy:
    \item Get the current position of Pacman from the game state.
    \item Get the list of legal actions that Pacman can take from the game state.
    \item Initialize an empty list to store the expected utility of each action.
    \item For each action, compute the successor position of Pacman if that action is taken.
    \item For each successor position, compute the expected utility by summing the product of the belief and the inverse of the maze distance for each ghost.
    \item Choose the action that maximizes the expected utility.
\end{itemize}

\subsection*{Mark the average score of the greedy strategy and of your alternative in your report.}

\section*{Question 14}
\subsection*{In both tests, pacman knows that the ghostswill move to the sides of the gameboard. What is different between the tests,and why?}

    Average score for greedy strategy: 754.700\\
    Average score for probabilistic approach: 763.000\\
    The reason for this difference could be due to the nature of the strategies.
    The greedy strategy only considers the most likely position of each ghost and chooses an action that minimizes the maze distance to the closest ghost.
    This can sometimes lead to suboptimal decisions, as it does not take into account the overall layout of the maze or the probability of the ghost being in each position.
    On the other hand, the alternative strategy takes into account the belief distributions of the ghosts' positions.
    It not only considers the most likely position of each ghost, but also the probability of the ghost being in each position.

\end{document}